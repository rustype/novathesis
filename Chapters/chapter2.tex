%!TEX root = ../template.tex
\chapter{Related Work}\label{cha:related-work}

\section{Language Preprocessors}\label{sec:lang-preprocessors}

Language preprocessors are a mechanism which runs during compilation,
some languages will apply the preprocessor during different compilation stages while others will only apply the preprocessor in a single stage.

% \subsection{C and C++}\label{sec:lang-preprocessors:c-cpp}

\subsection{OCaml}\label{sec:lang-preprocessors:ocaml}

The OCaml ecosystem currently uses OCaml \gls{PPX},
however, previous to version 4.02, OCaml made use of \gls{p4}.

We briefly review both \gls{p4} and \gls{PPX}.

\subsubsection*{Camlp4}\label{sec:lang-preprocessors:ocaml:p4}

Camlp4 is a parsing library which provides extensible grammars,
its main goal is to allow users to extend OCaml syntax,
Camlp4 is also able to redefine the core syntax,
OCaml even introduced a revised syntax\footnote{\url{https://caml.inria.fr/pub/docs/manual-camlp4/manual007.html}} to enable Camlp4.

The library has been deprecated due to being confusing to users and tools alike.
Users were required to learn the revised OCaml syntax which complicates the development process.
These criticisms are found throughout documents which discuss Camlp4\footnote{\url{https://whitequark.org/blog/2014/04/16/a-guide-to-extension-points-in-ocaml/}}.

In a nutshell, the Camlp4 library would allow developers to develop an extension syntax,
when the compiler would pass the source code as text to the preprocessor,
which, in turn would generate valid OCaml source code.

\subsubsection*{PPX}\label{sec:lang-preprocessors:ocaml:ppx}

\subsection{Java}\label{sec:lang-preprocessors:java}

As other languages, Java is also capable of source code processing during compile time,
we review two existing approaches, annotations and the ExtendJ compiler.

\subsubsection*{Java Annotation Processor}\label{sec:lang-preprocessors:java:annotation}

Java annotations were first introduced in Java 5 (\href{https://jcp.org/en/jsr/detail?id=269}{JSR 269}),
they are a form of metadata which can be added to Java source code.
Annotations can be used in conjunction with several components of the Java language,
such as classes, interfaces, documentation and others.
These are processed by build-time tools or by run-time libraries to achieve new semantic effects,
a popular example of such library would be the compile-time dependency injection framework \href{https://dagger.dev/}{Dagger 2}.

\paragraph{Annotation Syntax} is simple: every annotation starts with an \texttt{@} and is followed by an identifier, optionally, the annotation can take parameters.

\begin{lstlisting}
@Annotation
\end{lstlisting}

\subsubsection*{ExtendJ \& JastAdd}\label{sec:lang-preprocessors:java:extendj}

\subsection{Kotlin}\label{sec:lang-preprocessors:kotlin}

\subsubsection*{Kotlin Compiler Plugins}\label{sec:lang-preprocessors:kotlin:annotation}

\subsection{Rust}\label{sec:lang-preprocessors:rust}
