%!TEX root = ../template.tex
\chapter{Background (03/02/2021)}\label{cha:background}

\section{Systems Programming Languages}\label{sec:systems-programming}

The definition of the term \emph{systems programming language} is not agreed upon,
being somewhat flexible and ever-changing due to constant shift in requirements for applications.

Before the cloud, in the age of C, a systems programming language would
most likely be a language able to provide an adequate interface between the programmer and the machine.
Nowadays, the definition is more vague, as machines and software grow in complexity,
and the definition of system grows from single computer to a distributed system,
interfacing with the hardware in a more direct fashion is mostly not required.
Systems programming languages become about being able to produce a standalone
binary able to run on a variety of machines without requiring extra software.

\subsection{C}

C is a general-purpose programming language, while it can be considered a high-level programming language
when put besides assembly, it also fits the description of a low-level level programming language when besides languages like Python.
It was originally designed by Dennis Ritchie for the PDP-11 and has been around since 1972 \autocite{Kernighan1988},
C is by no means modern, being older than myself and most likely to outlive me.

Designed in a different time, its mental model is also different, the language is simple and straight forward,
the designers had goals to achieve and designed the language with them in mind.
Such mentality is noticeable when using the language,
it is simple as the hardware was and the level of control C provides is unparalleled,
being both a major benefit and a hindrance.
An expert programmer is able to take advantage of the language to produce highly-efficient software,
but a novice programmer will often find himself battling memory and pointer management bugs.

Its influence echoes in the modern languages,
whether in the form of syntax (i.e. the famous C-style syntax) or in the problems it tries to solve.
Languages such as Java take from C their syntax as well as one problem to solve, memory management;
other languages like Julia \autocite{Bezanson2017} aim to mimic its performance.

While not as popular as other languages, C was able to keep its relevance in the modern development landscape,
some of the most used software in the world is either written or powered by C.
The Linux kernel, which powers servers, the world's most powerful computers
and serves as a base for Android and other mobile devices,
git, Redis and nginx are also software examples which reached the top of their respective fields.

% While C and C++ constitute two very different languages,
% however their goals are relatively while C stayed relatively close to its origin through the years, C++ has grown in size.
% The biggest companies on earth, such as Google, Amazon and Microsoft have large codebases written in C++,
% most operating systems are written either in C or C++ and
% industries which require highly-performant software are powered by one of the two, or both.

% While both are in widespread use, they are not without their problems,
% in particular, C++ is known for its long compile times,
% a problem acknowledged by its creator Bjarne Stroustrup \autocite{Torre2014}.
% In some regards, C is considered too old, not having “modern” language features which would help developers (e.g. generics).
% Finally, both suffer from the memory management problem, leaving it up to the programmer,
% which inevitably leads to problems caused by human error \autocite{chromium, Miller2019}.

\subsection{C++}

Introduced in 1985 as an extension to C; its author, Bjarne Stroustrup writes:

\begin{displayquote}[{\autocite[Section 1.2]{Stroustrup1986}}]
    C++  is  based  on  the idea of providing both:
    \begin{compactitem}
        \item direct mappings of built-in operations and types to hardware to provide efficient memory use and efficient low-level operations, and
        \item affordable and flexible abstraction mechanisms to provide user-defined types with the same notational support, range of uses, and performance as built-in types.
    \end{compactitem}
\end{displayquote}

The language has since gone on to conquer the programming world, being used in a wide variety of software and hardware.
Currently, companies such as Google, Amazon and Microsoft have widespread adoption of C++ in their codebases.
Industries requiring the best performance as possible of the host, such as scientific computing,
financial software, AAA games and visual effects will most likely be running C++.

Just like C, C++ has its problems.
The language is enormous, with very complicated parts (e.g. templates) and compilation for big projects is very slow,
the author acknowledges this in \autocite{Torre2014}.
Furthermore, as the language provides a high level of control over the system, it has manual memory management,
suffering from the same problems as C.
Even with smart pointers (e.g. \texttt{unique\_ptr}) the problem is not considered solved,
as they introduce overhead in the most demanding applications.


\subsection{Ada}

Ada was developed in a standardization effort for USA's Department of Defense,
unifying projects spanning over 450 programming languages \autocite{Ada2021}.
Its main focus was the development of embedded applications,
currently the Ada language is mostly used in the critical domain due to its strong emphasis on safety,
some Ada success stories are the London Metro Victoria Line and the Paris Metro Line \autocite{SIGAda2021}.
Ada is also used in several other domains, such as aviation, space vehicles, financial systems and more \autocite{Feldman2014}.

\subsection{Go}

The Go programming language (or \texttt{golang}) is a Google project,
according to its tale, it was designed by the authors while they waited for their C++ code to compile.
Go tried to address several of the criticisms to C, namely memory management,
which it solved by using a garbage collector.
While it has made a name for itself in the network and distributed systems sector,
being the main language behind projects like Docker \autocite{Docker2021} and Kubernetes \autocite{Kubernetes2021},
its position as a systems programming language can be discussed.

Sometimes, however, the performance might not be enough,
as was the case for Discord, the popular internet voice server company, as demand increased,
Go was not able to meet the expected performance requirements and the company replaced it with Rust \autocite{Howarth2020}.
In \autocite{Torre2014}, one of Go's authors, Rob Pike, says that he regrets categorizing Go as a systems programming language,
being rather a server programming language that evolved into a cloud infrastructure language.
Regardless of discussion, Go has proven to be a viable alternative to its counterparts,
compromising extreme performance in name of safety and simplicity.

\section{The Rust Language}\label{sec:rust-lang}

Rust is a fairly recent systems programming language,
it started as a side project of the author Graydon Hoare,
its public history dates back to 2010 \autocite{Hoare2010}.
In 2012 Mozilla picked up Rust to help develop the Servo browser engine, the successor to the previous Gecko engine;
as a way to test Rust's capabilities \autocite{Klabnik2016}.

\subsection{What makes Rust different?}

In comparison with other languages, one of the first things someone new to Rust ought to notice is the emphasis put on safety.
Being a competitor to C++ rather than higher-level languages like Java,
while still providing C++-level performance and memory safety is an achievement in of itself.
Rust, however, also aims to allow users to be productive without sacrificing safety or performance.

The key to all the promises Rust makes lies in its ownership system and the borrow checker,
completely new mechanisms when compared with other mainstream languages,
but a product of years of research both in academia and the industry.
This mechanism merits most of Rust's accomplishments and also its biggest problem, the learning experience.
While Rust has become more accessible over the years,
ownership and the borrow checker still require some effort on the part of the developer to learn.
I provide a small overview of ownership, the borrow checker and their part in Rust's promise of “\emph{fearless concurrency}”.


\subsection{Ownership}\label{sec:rust-lang:ownership}

Ownership is the mechanism used by Rust to ensure no memory block stays allocated longer than it is required to.
Through ownership, the compiler is able to free memory when required,
inserting the respective deallocation calls in the output program.
Behind ownership, there are three rules:

\begin{displayquote}[{\autocite[Section 4.1]{RustBook2021}}]
    \begin{compactitem}
        \item Each value in Rust has a variable that’s called its owner.
        \item There can only be one owner at a time.
        \item When the owner goes out of scope, the value will be dropped.
    \end{compactitem}
\end{displayquote}

To illustrate the rules, consider \autoref{lst:rust-move}, where we have two variables \texttt{x} and \texttt{y}.
First, \texttt{"Hello"}\footnotemark is assigned to \texttt{x}, thus \texttt{x} now owns \texttt{"Hello"}.
After, \texttt{x} is assigned to \texttt{y}, consider the second rule of ownership, since we can only have one owner,
\texttt{x}'s value ownership is transferred to \texttt{y}.
Since we transferred \texttt{x}'s value to \texttt{y}, \texttt{x} is no longer valid, consequently,
when compiling the code an error will be issued due to \texttt{x} being moved.

Notice how \texttt{String::from} is used instead of another type,
since \keyword{String} type does not implement \keyword{Copy} it can only be moved.
If the used type implemented \keyword{Copy}, the value would have been copied instead of moved.

\begin{listing}
    \begin{minted}{Rust}
let x = String::from("Hello");  // ok: `x` is assigned "Hello"
let y = x;                      // ok: `x` is moved into `y`
println!("{}", x);              // error: `x` was moved in the previous line
    \end{minted}
    \caption{Example of the move-by-default mechanism to enforce ownership.}
    \label{lst:rust-move}
\end{listing}

So far this illustrates the first two rules. The last rule can be considered invisible,
as it happens during compilation and the user would not notice it usually.
What happens is that at the end of the scope, any variable whose owner is in scope, will be freed.
While the developer is not required to explicitly free the memory, the compiler will insert the calls for the developer.

\subsection{Borrowing}\label{sec:rust-lang:borrowing}

If the developer could only copy or move memory the usability of the language would be severely limited.
For example, functions that read a variable and produce a new value,
not requiring the variable to be consumed would be impossible.
To cope with this, Rust allows values to be \emph{borrowed}, in other words,
the owner of the variable allows for it to be read by others.

To borrow a value, one writes \texttt{\&value}, this creates a read-only reference to \texttt{value}.
There can be an unlimited number of read-only references to a value, but only a single mutable reference.
This is discussed in \autoref{sec:rust-lang:concurrency}. Consider the example \autoref{lst:rust-borrow}.
In the example, \texttt{x} is now possible to be printed since it was not moved into \texttt{y}.
Rather, \texttt{y} borrowed \texttt{x} through a reference.

Going back to the rules (\autoref{sec:rust-lang:ownership}), Rust's references obey them as all other values,
the variable containing them has ownership \emph{over the reference}; it still is a single owner,
as if \mintinline{Rust}{let z = y;} was to be added, the reference would be copied instead of moved;
and finally, when the owner goes out of scope \emph{the reference is dropped}, but not original the value.

\begin{listing}
    \begin{minted}{Rust}
let x = String::from("Hello");  // ok: `x` is assigned "Hello"
let y = &x;                     // ok: `x` is borrowed to `y`
println!("{}", x);              // ok: `x` can be printed since it is still valid
    \end{minted}
    \caption{Example using borrowing to allow for more than one reader on the same variable.}
    \label{lst:rust-borrow}
\end{listing}

\subsubsection*{Mutable Borrows}

One last thing to consider are mutable borrows.
As previously discussed, in Rust it is possible to create multiple immutable references but only one mutable reference.
Regarding mutable references there are two cases to consider:
\paragraph{$N$ mutable references,} see \autoref{lst:rust-borrow-n-mut}.
Understanding why only one mutable reference can exist at a time is trivial,
as multiple mutable references to the same object would allow it to be mutated concurrently,
which could lead to inconsistent values.

\paragraph{$N$ immutable references and $1$ mutable reference,} see \autoref{lst:rust-borrow-n-immut-1-mut}.
The reason behind not allowing a mutable reference to coexist is similar.
Consider that each value can be executed by a different thread,
the first two (\texttt{r1} and \texttt{r2}) are only read and the third (\texttt{r3}) can be read and written.
While there will be no conflicts between writers, it is possible for the readers to read an inconsistent value,
since it can happen during the write operation.

\begin{listing}
    \begin{minted}{Rust}
let mut s = String::from("hello");
let r1 = &mut s; // ok: first mutable borrow
let r2 = &mut s; // error: `s` was mutably borrowed in the previous line
    \end{minted}
    \caption{Example using borrowing to allow for more than one reader on the same variable.}
    \label{lst:rust-borrow-n-mut}
\end{listing}

\begin{listing}
    \begin{minted}{Rust}
let mut s = String::from("hello");
let r1 = &s;     // ok: first immutable borrow
let r2 = &s;     // ok: second immutable borrow
let r3 = &mut s; // error: `s` was immutable borrowed in the previous lines
    \end{minted}
    \caption{Example using borrowing to allow for more than one reader on the same variable.}
    \label{lst:rust-borrow-n-immut-1-mut}
\end{listing}


\subsection{Concurrency}\label{sec:rust-lang:concurrency}

\begin{displayquote}[{\autocite[Section 16]{RustBook2021}}]
    Initially, the Rust team thought that ensuring memory safety and preventing concurrency problems were two separate challenges to be solved with different methods.
    Over time, the team discovered that the ownership and type systems are a powerful set of tools to help manage memory safety and concurrency problems!
    By leveraging ownership and type checking, many concurrency errors are compile-time errors in Rust rather than runtime errors.
\end{displayquote}

Rust provides several kinds of mechanisms to prevent concurrency related problems.
Mechanisms as \emph{message-passing}, \emph{shared-state} and
traits to enable developers to extend upon the existing abstractions.

\subsubsection*{Message-passing}

Rust's message-passing library is inspired on Go's approach to concurrency,
prioritizing message passing over other kinds of concurrent approaches, such as locking.

\begin{displayquote}[{\autocite[Concurrency]{Go2021}}]
    Do not communicate by sharing memory; instead, share memory by communicating.
\end{displayquote}

Rust defines channels which have two ends, the transmitter and the receiver.
The former can also be seen as the sender, and when is declared with the message type,
the latter is also declared with the message type, they can be the same or distinct.

The ownership system comes in when the transmitter sends a message,
when received the ownership of the message is taken on by the receiver end.
This enforces that values cannot be in both sides of the communication at the same time,
preventing concurrent accesses.

\subsubsection*{Shared-state}

Along with message-passing, Rust allows memory to be shared in a concurrent, safe way.
Just as before, Rust's ownership system also helps with mutexes' biggest problem, locking and unlocking.

In a language like Java, whenever a thread is able to call \texttt{lock} on a mutex,
it is required to call \texttt{unlock} on it, only then can other threads can use it.
The problem is that this approach is subject to human error,
forgetting to call unlock or calling unlock in the wrong place is possible.
Making use of the ownership system, Rust is able to know when the lock reached the end of the scope and should be dropped.

\subsection{Why Rust instead of Language X?}

% i really don't think the billion dollar mistake is a valid point here
% if the maybe monad cant be considered a typestate, the mistake cant really be considered as a valid argument

The main obstacle between typestates and programming languages is the requirement for aliasing control.
In short, typestates are incompatible with aliasing (details are provided in \autoref{sec:typestates}).

As discussed in \autoref{sec:rust-lang:borrowing}, Rust's ownership system allows for aliasing control.
Using moves to enforce the consumption of values,
immutable references for pure functions and mutable ones for limited mutability,
it is possible to emulate typestates.


\section{Behavioral Types}\label{sec:behavioral-types}

\subsection{Session Types}

\subsection{Typestates}\label{sec:typestates}

\begin{displayquote}[{\autocite{Strom1983}}]
    ... traditional strong type checking was enhanced with \textbf{typestate checking}
    a new mechanism in which the compiler guarantees that for all execution paths,
    the sequence of operations on each variable obeys a finite state grammar associated with that variable's type.
\end{displayquote}

The first language to make use of typestates was NIL~\autocite{Strom1983},
afterwards languages like Hermes~\autocite{Strom1990} and Plaid~\autocite{Aldrich2009}
extended the concept with new techniques.

\subsubsection*{The case for typestates}

As discussed in \autoref{sec:context}, bugs in systems programming are costly,
thus, bugs must be minimized.
Several tools, such as static analyzers, fuzzers, testing frameworks and others,
aid in this purpose, if we have all these external tools,
why should we not try and leverage the programming language itself?

\paragraph{Moving towards better languages.}
Programming languages allow the programmer to express a set of actions to be taken by the computer,
they are tools which enable us to achieve a goal.
Being essential to our work, better tools enable developers to be more productive and achieve higher quality work.
The remaining question is “\emph{why do we not create better languages?}”.
Even when considering languages to be cheap to develop,
the amount of work between a \emph{working} language to be \emph{production ready} is not cheap.
Furthermore, while adopting a new language for a hobby project is easy,
the same does not apply for enterprise level projects,
requiring several developers to know the ins and outs of the language.

\paragraph{Static typed languages.}
The current trend is to move from dynamically typed languages,
to statically typed ones, or at the very least, add typing support to existing dynamic languages.
Typescript~\autocite{typescript},
Reason~\autocite{reason} and
PureScript~\autocite{purescript}
are all examples of languages built to bridge the gap between static type systems and JavaScript.
Python and Ruby, two popular dynamic languages, have also pushed for type adoption
with the addition of type hint support in recent
releases~\autocite{PythonTyping, RubyRBS}.

\paragraph{Where do typestates fit?}
Typestates are a complex subject, able to be adopted at several levels,
just like type hints, they can be partially used in some languages,
through tools such as Mungo~\autocite{Voinea2020},
by contract-style assertions as in Ada2012, Eiffel or pre-0.4 Rust,
or finally by leveraging the existing type system to write typestate enabled code as it is possible in
Rust~\autocite{Duarte2020}.

\paragraph{Why use typestates?}
By leveraging the state to the typesystem, the compiler is able to aid the programmer during development,
a given set of transitions will be impossible by default, since the types do not implement them. % TODO maybe add an example
By reducing the need for developers to check for a certain set of conditions through the use of typestates,
it becomes possible to reduce the number of runtime assertions and
completely eliminate the need for illegal state exceptions since illegal transitions are checked at compile time.


\subsubsection*{Typestates in action}

As a simple example, consider the Java application in \autoref{fig:java-mult} which simply takes two numbers and multiples them together.
The application will throw an exception on line 6,
since the programmer closed the \texttt{Scanner} in line 5.
In this example, the error is simple to catch,
the program is short and the \texttt{Scanner} can either be open or closed,
however, real-world applications are not that simple.

\begin{listing}
    \centering
    \begin{minted}{java}
public class Mult {
    public static void main(String[] args) {
        Scanner s = new Scanner(System.in);
        s.nextLine();
        s.close();
        s.nextLine();
    }
}
    \end{minted}
    \caption{The \texttt{Mult} program, which reads two integer and multiplies them together.}
    \label{fig:java-mult}
\end{listing}

In the case of \emph{typestated} programming,
the type system will provide the programmer with better tools to express state,
furthermore, the compiler will then catch errors regarding state,
such as the previous \emph{use-after-close}.

\autoref{fig:java-mult-typestate} shows the \texttt{Mult} program written in a typestated fashion,
notice that the \texttt{Scanner} type is now augmented with its state and
the compiler is able to catch the misuse of the \texttt{Scanner[Closed]} interface.


\begin{listing}
    \centering
    \begin{minted}{java}
public class Mult {
    public static void main(String[] args) {
        Scanner[Open] s = new Scanner(System.in);
        s.nextLine();
        Scanner[Closed] s = s.close();
        s.nextLine(); // compiler error
    }
}
    \end{minted}
    \caption{The \texttt{Mult} program, written in a typestated fashion.}
    \label{fig:java-mult-typestate}
\end{listing}

\paragraph{Plaid} is a typestate-oriented programming language \autocite{Aldrich2009},
instead of \keyword{class}es users write \keyword{typestate}s.
Each typestate represents a class in its possible states,
its methods and behavior change during runtime as state changes,
in contrast with other languages (e.g. Java) where public methods and fields are always available.

This property allows the typesystem to enforce certain properties at compile time,
such as certain methods will never be called in a given state since it is not possible by design
(i.e. they are not available in the interface).


\paragraph{Rust.}
As discussed in \autoref{sec:rust-lang}, Rust takes its commitment with safety with seriousness,
providing the necessary tools to users.
While Rust does not support first-class typestates,
it is possible to emulate them using its type system (as demonstrated in \autocite{Duarte2020}),
this is discussed in further sections of this document.

\begin{description}
    \item[Embedded Rust.] As any systems programming language, Rust penetrated the embedded development space.
          Its features are most adequate and the community has put great effort into making Rust a viable language for embedded systems.

          \emph{The Embedded Rust Book}'s~\autocite{Rust2021} Chapter 4 is dedicated to static guarantees,
          introducing programmers to the concepts of typestate in Section 4.1, and their usage in embedded systems.

          As for real-world usage, typestates are abundantly used in the area (not just discussed in the book),
          under \autocite{Stm32} one finds several repositories (suffixed with \texttt{-hal})
          which implement typestates
          (e.g. \href{https://github.com/stm32-rs/stm32h7xx-hal/blob/master/src/gpio.rs#L51-L128}{\texttt{gpio.rs}}
          from \texttt{stm32h7xx-hal}).
\end{description}

\paragraph{Obsidian} is a language targeting Hyperledger Fabric \autocite{Fabric2021},
among other features it makes use of typestates to reduce the amount of bugs when dealing with assets.

In \autocite{Coblenz2020} an empirical study tested and proved Obsidian claims,
when compared with Solidity, the leading blockchain language,
users inserted fewer bugs and were able to start developing safer code faster.

