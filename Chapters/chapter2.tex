%!TEX root = ../template.tex
\chapter{Background (03/02/2021)}\label{cha:background}

\section{Systems Programming Languages}\label{sec:systems-programming}

The definition of the term \emph{systems programming language} is not agreed upon,
being somewhat flexible and ever-changing due to constant shift in requirements for applications.

Before the cloud, in the age of C, a systems programming language would
most likely be a language able to provide an adequate interface between the programmer and the machine.
Nowadays, the definition is more vague, as machines and software grow in complexity,
and the definition of system grows from single computer to a distributed system,
interfacing with the hardware in a more direct fashion is mostly not required.
Systems programming languages become about being able to produce a standalone
binary able to run on a variety of machines without requiring extra software.

\subsection{C}

C is a general-purpose programming language, while it can be considered a high-level programming language
when put besides assembly, it also fits the description of a low-level level programming language when besides languages like Python.
It was originally designed by Dennis Ritchie for the PDP-11 and has been around since 1972 \autocite{Kernighan1988},
C is by no means modern, being older than myself and most likely to outlive me.

Designed in a different time, its mental model is also different, the language is simple and straight forward,
the designers had goals to achieve and designed the language with them in mind.
Such mentality is noticeable when using the language,
it is simple as the hardware was and the level of control C provides is unparalleled,
being both a major benefit and a hindrance.
An expert programmer is able to take advantage of the language to produce highly-efficient software,
but a novice programmer will often find himself battling memory and pointer management bugs.

Its influence echoes in the modern languages,
whether in the form of syntax (i.e. the famous C-style syntax) or in the problems it tries to solve.
Languages such as Java take from C their syntax as well as one problem to solve, memory management;
other languages like Julia \autocite{Bezanson2017} aim to mimic its performance.

While not as popular as other languages, C was able to keep its relevance in the modern development landscape,
some of the most used software in the world is either written or powered by C.
The Linux kernel, which powers servers, the world's most powerful computers
and serves as a base for Android and other mobile devices,
git, Redis and nginx are also software examples which reached the top of their respective fields.

% While C and C++ constitute two very different languages,
% however their goals are relatively while C stayed relatively close to its origin through the years, C++ has grown in size.
% The biggest companies on earth, such as Google, Amazon and Microsoft have large codebases written in C++,
% most operating systems are written either in C or C++ and
% industries which require highly-performant software are powered by one of the two, or both.

% While both are in widespread use, they are not without their problems,
% in particular, C++ is known for its long compile times,
% a problem acknowledged by its creator Bjarne Stroustrup \autocite{Torre2014}.
% In some regards, C is considered too old, not having “modern” language features which would help developers (e.g. generics).
% Finally, both suffer from the memory management problem, leaving it up to the programmer,
% which inevitably leads to problems caused by human error \autocite{chromium, Miller2019}.

\subsection{C++}

Introduced in 1985 as an extension to C; its author, Bjarne Stroustrup writes:

\begin{displayquote}[{\autocite[Section 1.2]{Stroustrup1986}}]
    C++  is  based  on  the idea of providing both:
    \begin{itemize}
        \item direct mappings of built-in operations and types to hardware to provide efficient memory use and efficient low-level operations, and
        \item affordable and flexible abstraction mechanisms to provide user-defined types with the same notational support, range of uses, and performance as built-in types.
    \end{itemize}
\end{displayquote}

The language has since gone on to conquer the programming world, being used in a wide variety of software and hardware.
Currently, companies such as Google, Amazon and Microsoft have widespread adoption of C++ in their codebases.
Industries requiring the best performance as possible of the host, such as scientific computing,
financial software, AAA games and visual effects will most likely be running C++.

Just like C, C++ has its problems.
The language is enormous, with very complicated parts (e.g. templates) and compilation for big projects is very slow,
the author acknowledges this in \autocite{Torre2014}.
Furthermore, as the language provides a high level of control over the system, it has manual memory management,
suffering from the same problems as C.
Even with smart pointers (e.g. \texttt{unique\_ptr}) the problem is not considered solved,
as they introduce overhead in the most demanding applications.


\subsection{Ada}

Ada was developed in a standardization effort for USA's Department of Defense,
unifying projects spanning over 450 programming languages \autocite{Ada2021}.
Its main focus was the development of embedded applications,
currently the Ada language is mostly used in the critical domain due to its strong emphasis on safety,
some Ada success stories are the London Metro Victoria Line and the Paris Metro Line \autocite{SIGAda2021}.
Ada is also used in several other domains, such as aviation, space vehicles, financial systems and more \autocite{Feldman2014}.

\subsection{Go}

The Go programming language (or \texttt{golang}) is a Google project,
according to its tale, it was designed by the authors while they waited for their C++ code to compile.
Go tried to address several of the criticisms to C, namely memory management,
which it solved by using a garbage collector.
While it has made a name for itself in the network and distributed systems sector,
being the main language behind projects like Docker \autocite{Docker2021} and Kubernetes \autocite{Kubernetes2021},
its position as a systems programming language can be discussed.

Sometimes, however, the performance might not be enough,
as was the case for Discord, the popular internet voice server company, as demand increased,
Go was not able to meet the expected performance requirements and the company replaced it with Rust \autocite{Howarth2020}.
In \autocite{Torre2014}, one of Go's authors, Rob Pike, says that he regrets categorizing Go as a systems programming language,
being rather a server programming language that evolved into a cloud infrastructure language.
Regardless of discussion, Go has proven to be a viable alternative to its counterparts,
compromising extreme performance in name of safety and simplicity.

\section{The Rust Language}\label{sec:rust-lang}

Rust is a fairly recent systems programming language,
its main focus revolves around memory safety,
effectively removing classes of such bugs (e.g. \emph{use-after-free} and \emph{double-free}).
Another one of Rust's focus, is on productivity,
aiming to provide the safety mechanisms necessary to remove the previous class of bugs,
while trying to provide a pleasant and productive development experience.

To achieve its goals, Rust makes use of a borrow checker and an ownership system,
in conjunction, they're responsible for guaranteeing correct memory usage.
At its core, the borrow checker is a lightweight theorem prover,
it tries to prove that the code does not break safety rules.

Its rules can be distilled down to the following intuition,
only one entity can hold a reference to mutable data at a time,
several entities may hold references to immutable data.
This also enables Rust to also provide mechanisms which help deal with concurrency,
allowing for developers to write data-race free
code~\autocite{Turon2015}.

\subsection{What makes Rust different?}

\section{Behavioral Types}\label{sec:behavioral-types}

\subsection{Session Types}


\subsection{Typestates}\label{sec:typestates}

\begin{displayquote}[{Typestates  described in \autocite{Strom1983}}]
    ... traditional strong type checking was enhanced with \textbf{typestate checking}
    a new mechanism in which the compiler guarantees that for all execution paths,
    the sequence of operations on each variable obeys a finite state grammar associated with that variable's type.
\end{displayquote}

The first language to make use of typestates was NIL~\autocite{Strom1983},
afterwards languages like Hermes~\autocite{Strom1990} and Plaid~\autocite{Aldrich2009}
extended the concept with new techniques.

\subsubsection*{The case for typestates}

As discussed in \autoref{sec:context}, bugs in systems programming are costly,
thus, bugs must be minimized.
Several tools, such as static analyzers, fuzzers, testing frameworks and others,
aid in this purpose, if we have all these external tools,
why should we not try and leverage the programming language itself?

\paragraph{Moving towards better languages.}
Programming languages allow the programmer to express a set of actions to be taken by the computer,
they are tools which enable us to achieve a goal.
Being essential to our work, better tools enable developers to be more productive and achieve higher quality work.
The remaining question is “\emph{why do we not create better languages?}”.
Even when considering languages to be cheap to develop,
the amount of work between a \emph{working} language to be \emph{production ready} is not cheap.
Furthermore, while adopting a new language for a hobby project is easy,
the same does not apply for enterprise level projects,
requiring several developers to know the ins and outs of the language.

\paragraph{Static typed languages.}
The current trend is to move from dynamically typed languages,
to statically typed ones, or at the very least, add typing support to existing dynamic languages.
Typescript~\autocite{typescript},
Reason~\autocite{reason} and
PureScript~\autocite{purescript}
are all examples of languages built to bridge the gap between static type systems and JavaScript.
Python and Ruby, two popular dynamic languages, have also pushed for type adoption
with the addition of type hint support in recent
releases~\autocite{PythonTyping, RubyRBS}.

\paragraph{Where do typestates fit?}
Typestates are a complex subject, able to be adopted at several levels,
just like type hints, they can be partially used in some languages,
through tools such as Mungo~\autocite{Voinea2020},
by contract-style assertions as in Ada2012, Eiffel or pre-0.4 Rust,
or finally by leveraging the existing type system to write typestate enabled code as it is possible in
Rust~\autocite{Duarte2020}.

\paragraph{Why use typestates?}
By leveraging the state to the typesystem, the compiler is able to aid the programmer during development,
a given set of transitions will be impossible by default, since the types do not implement them. % TODO maybe add an example
By reducing the need for developers to check for a certain set of conditions through the use of typestates,
it becomes possible to reduce the number of runtime assertions and
completely eliminate the need for illegal state exceptions since illegal transitions are checked at compile time.


\subsubsection*{Typestates in action}

As a simple example, consider the Java application in \autoref{fig:java-mult} which simply takes two numbers and multiples them together.
The application will throw an exception on line 6,
since the programmer closed the \texttt{Scanner} in line 5.
In this example, the error is simple to catch,
the program is short and the \texttt{Scanner} can either be open or closed,
however, real-world applications are not that simple.

\begin{listing}
    \centering
    \begin{minted}{java}
public class Mult {
    public static void main(String[] args) {
        Scanner s = new Scanner(System.in);
        s.nextLine();
        s.close();
        s.nextLine();
    }
}
    \end{minted}
    \caption{The \texttt{Mult} program, which reads two integer and multiplies them together.}
    \label{fig:java-mult}
\end{listing}

In the case of \emph{typestated} programming,
the type system will provide the programmer with better tools to express state,
furthermore, the compiler will then catch errors regarding state,
such as the previous \emph{use-after-close}.

\autoref{fig:java-mult-typestate} shows the \texttt{Mult} program written in a typestated fashion,
notice that the \texttt{Scanner} type is now augmented with its state and
the compiler is able to catch the misuse of the \texttt{Scanner[Closed]} interface.


\begin{listing}
    \centering
    \begin{minted}{java}
public class Mult {
    public static void main(String[] args) {
        Scanner[Open] s = new Scanner(System.in);
        s.nextLine();
        Scanner[Closed] s = s.close();
        s.nextLine(); // compiler error
    }
}
    \end{minted}
    \caption{The \texttt{Mult} program, written in a typestated fashion.}
    \label{fig:java-mult-typestate}
\end{listing}

\paragraph{Plaid} is a typestate-oriented programming language \autocite{Aldrich2009},
instead of \keyword{class}es users write \keyword{typestate}s.
Each typestate represents a class in its possible states,
its methods and behavior change during runtime as state changes,
in contrast with other languages (e.g. Java) where public methods and fields are always available.

This property allows the typesystem to enforce certain properties at compile time,
such as certain methods will never be called in a given state since it is not possible by design
(i.e. they are not available in the interface).


\paragraph{Rust.}
As discussed in \autoref{sec:rust-lang}, Rust takes its commitment with safety with seriousness,
providing the necessary tools to users.
While Rust does not support first-class typestates,
it is possible to emulate them using its type system (as demonstrated in \autocite{Duarte2020}),
this is discussed in further sections of this document.

\begin{description}
    \item[Embedded Rust.] As any systems programming language, Rust penetrated the embedded development space.
          Its features are most adequate and the community has put great effort into making Rust a viable language for embedded systems.

          \emph{The Embedded Rust Book}'s~\autocite{Rust2021} Chapter 4 is dedicated to static guarantees,
          introducing programmers to the concepts of typestate in Section 4.1, and their usage in embedded systems.

          As for real-world usage, typestates are abundantly used in the area (not just discussed in the book),
          under \autocite{Stm32} one finds several repositories (suffixed with \texttt{-hal})
          which implement typestates
          (e.g. \href{https://github.com/stm32-rs/stm32h7xx-hal/blob/master/src/gpio.rs#L51-L128}{\texttt{gpio.rs}}
          from \texttt{stm32h7xx-hal}).
\end{description}

\paragraph{Obsidian} is a language targeting Hyperledger Fabric \autocite{Fabric2021},
among other features it makes use of typestates to reduce the amount of bugs when dealing with assets.

In \autocite{Coblenz2020} an empirical study tested and proved Obsidian claims,
when compared with Solidity, the leading blockchain language,
users inserted fewer bugs and were able to start developing safer code faster.

