%!TEX root = ../template.tex
\chapter{Contributions \& Planning}\label{cha:planning}

\section{Objectives}

As previously discussed, the present work proposes the design of a new DSL embedded in Rust.
The core goal of the DSL is to provide typestates without getting in the way of the developer.
This means providing a syntax that stays close to Rust, only introducing new constructs when required.
Errors should be helpful and point to the original DSL code (see \autoref{lst:dsl-typestate-error})
, avoiding scenarios like the “\emph{infamous}” C++ template errors.
Features should be well documented allowing anyone to get up and running quickly,
along this idea, the language itself should be a simple dependency and avoid any complicated build process.

Finally, as discussed in \autoref{sec:objectives}, the language should provide static guarantees ---
such as state usefulness and automata minimality.
That is, respectively, if a state is reachable from the start state and if it is possible to reach the end state from it;
and if the automata is minimal.
Along with usefulness checks, the presence of start and final states are also checked,
otherwise such check would not work.

\begin{align}
    Start \rightarrow A~   & ;~A \rightarrow End   \\
    Start \rightarrow End~ & ;~Start \rightarrow B \\
    Start \rightarrow End~ & ;~B \rightarrow End
\end{align}

As an example, consider the previous equations as simple state machines.
From the three, the last two would not compile as it is impossible to reach $End$ from $B$, and it is impossible to reach $B$ from $Start$, respectively.

\section{Architecture}\label{sec:arch}

The DSL should provide a specification language,
allowing the developer to describe a structure, along with possible states and functions ---
which can be pure, impure (i.e. mutating some aspect of the structure without transitioning states)
or transitions (i.e. making the transition between states).
The specification is then transformed into the necessary Rust structures and traits.
In the end, the developer should only be required to write the \keyword{impl}s for each state.

\subsection{Selecting the correct macro}

In short, there are two valid candidates, function-like macros and attribute macros.
From the two I selected attribute macros,
not only they are easier to parse (since I can leverage existing libraries),
they are also close to Fugue \autocite{DeLine2004}.
The latter provides validation over the overall approach, as Fugue constitutes a tool developed in an industrial environment, for real world software.

\paragraph{Derive macros} seem to be a great candidate for our purpose.
We can add them to an enumeration (just like \autocite{Fitzgerald2019}) and derive the necessary states from there.
The main limitation of this approach is the lack of transparency —
the derived methods are fixed by the macro and the developer does not have control over them.

\paragraph{Attribute macros} Initially, I thought that leveraging attribute macros would not work.
Macro expansion does not keep state hence I could not relate several attributes together.
However, attribute macros are attachable to modules and inside them, one can write any kind of Rust code.
This approach allows full control over the module token stream, thus enabling processing and analysis over user code.

\paragraph{Function-like macros} enable expressive DSLs.
Their input can be \emph{almost} arbitrary and as previously discussed it is even possible to sketch a full language inside them.
In comparison with attribute macros they allow for more flexibility, however this may come at the cost of requiring the user to learn a new language, even if small.


\subsection{How does it work?}

As stated, this approach provides a DSL.
Such DSL is responsible for allowing the developer to write an expressive typestate specification.
The DSL processing is done as follows (illustrated in \autoref{fig:dsl-processing}):
\begin{compactenum}
    \item The DSL is first parsed during macro expansion.
    \item The extracted structure is converted into a state-machine.
    \item The state-machine is verified against our checks (see \autoref{sec:arch:fsm}).
    \item Extract all required \keyword{trait}s and \keyword{struct}s from the specification.
    \item Expand the macro into valid Rust code.
\end{compactenum}
After this process, all that is left to do is the implementation of the \keyword{trait}s.

\input{Chapters/Figures/C4/dsl-processing.tex}

\subsection{Syntax}\label{sec:arch:syntax}

% One of the DSL's syntax goals is to stay as close to Rust as possible.
The macro is invoked by attaching the \keyword{\#[typestate]} attribute to a module declaration.
This section reviews some important aspects, such as state usage and possible enhancements to function declarations.
The syntax itself is the same as Rust and can be seen in \autoref{lst:dsl-typestate-spec}.

\paragraph{The module declaration} is attached with \keyword{\#[typestate]}, this is the macro entrypoint (line 1 of \autoref{lst:dsl-typestate-spec}).
This syntax separates different typestates into individual modules.
The restriction can be lifted by adding a macro parameter that does not include \keyword{mod} in the final expansion.

\paragraph{The main structure} is declared inline with all others, the key difference being the lack of attributes (lines 2-4 of \autoref{lst:dsl-typestate-spec}).
The structure could be omitted in the case that no variables are required in all states,
the name could then be assumed to be the one in the module, or also given as parameter to the macro.

\paragraph{A state declaration} uses the \keyword{\#[state]} attribute (lines 3-11 of \autoref{lst:dsl-typestate-spec}).
This makes it clear that each marked structure maps to a state in the state machine.

\paragraph{A function declaration} in regular Rust, without attributes (lines 14-20 of \autoref{lst:dsl-typestate-spec}).
They can be one of three possible categories:
\begin{compactitem}
    \item \keyword{pure} functions, which do not mutate state (e.g. line 14 of \autoref{lst:dsl-typestate-spec}).
    Their signature takes an immutable reference to \keyword{self} as an argument,
    disabling mutation.
    \item \keyword{impure} functions, which can mutate state but not transition between states (e.g. line 15 of \autoref{lst:dsl-typestate-spec}).
    Their signature takes a mutable reference to \keyword{self} as an argument.
    This allows for mutation of the current state, but not transition.
    \item \keyword{transition} functions, which transition between states (lines 16-20 of \autoref{lst:dsl-typestate-spec}).
    Their signature takes ownership of \keyword{self},
    forcing the consumption of the current state disabling possible aliasing.
\end{compactitem}
These attributes can be explicit, but they can also be inferred from the code.
The reader may have noticed two details in \autoref{lst:dsl-typestate-spec} ---
the \texttt{Flying} state is not used,
the \texttt{fly\_to} function returns the same state as \keyword{self}.
This is the case since the function will make the transition between \texttt{Hovering}, \texttt{Flying} and back internally.
Thus requiring \keyword{self} to be consumed.



\paragraph{Initial and final states} are special cases as they are defined with both structures and functions.
The syntax to declare initial and final states is presented in lines 17-18 of \autoref{lst:dsl-typestate-spec}.
As the reader can observe, the syntax uses functions to declare with states are initial and final.
They can be declared explicitly or be inferred using the following rules:
\begin{compactitem}
    \item \textbf{Initial states} are declared with functions that do not take \keyword{self} as a parameter and return a state.
    The \keyword{self} keyword is omitted since the state is yet to be built.
    \item \textbf{Final states} are declared as “\emph{consumers}”, that is, they take \keyword{self} as a parameter and do not return a value.
    This syntax makes clear that the state becomes unavailable for further processing.
\end{compactitem}

\paragraph{Non-deterministic transitions} are useful to cover cases where the output is dependent on some external factors.
Consider a process that can fail, the developer is required to model both the success and failure cases.
To do so, the developer can declare an \keyword{enum}, containing possible outcomes.
Such enumeration has one key difference from regular Rust enumerations, the fields must contain existing states only,
that is, structures declared inside the same macro call (lines 8-11 and 15 of \autoref{lst:dsl-typestate-spec}).
Enforcing these transitions can either be done by leveraging Rust's enumerations and leaving the matching up to the user,
or through determinization of the automata.

\input{Chapters/Listings/C4/dsl-typestate-spec.tex}
\input{Chapters/Listings/C4/dsl-typestate-generated.tex}
\input{Chapters/Listings/C4/dsl-typestate-impl.tex}
\input{Chapters/Listings/C4/dsl-typestate-error.tex}

\subsection{State-Machine Handling}\label{sec:arch:fsm}

\paragraph{Conversion.}
The reader might have noticed that the specification syntax cleanly maps to a state machine.
As discussed in \autoref{sec:arch:syntax}, each \keyword{struct} is a state and each function is either a transition or a function which does not change state.
From the code, we can establish states as nodes and functions as edges,
effectively converting the specification into automata.

\paragraph{Verification.}
After extraction the state-machine is subject to verification.
This process is done using existing algorithms \autocite{Hopcroft2013}, to check the following properties:
\begin{compactitem}
    \item \textbf{Minimality.} The state machine should be minimal in the sense that there should be no redundancy in states and transitions.
    \item \textbf{Non-empty language.} The language of the state machine should not be empty; thus we need to ensure the presence of final states.
    \item \textbf{Usefulness.} As discussed, all states should be reachable from the start state and be able to reach the end state.
\end{compactitem}
In this case, the user should get feedback over any infraction,
explaining the reasoning behind the error or warning.

\subsection{Developer Workflow}
The final user workflow should be as follows:
\begin{compactenum}
    \item The developer starts by writing the specification, as in \autoref{lst:dsl-typestate-spec}.
    \item When done, the developer compiles the code.
    In turn, the compiler will begin the process of expanding the macro.
    This process is illustrated in \autoref{fig:dsl-processing}.
    \begin{compactitem}
        \item During expansion, if the state machine fails any of the verifications described in \autoref{sec:arch:fsm},
        a compiler error is issued.
        \item If no errors are reported, the process continues as normal and generates all required boilerplate, like in \autoref{lst:dsl-typestate-generated}.
        This boilerplate does not appear for the user, hence it is required the generation process is as predictable as possible.
    \end{compactitem}
    \item After generation, the user is now required to implement all functions, as in \autoref{lst:dsl-typestate-impl}.
\end{compactenum}


\section{Previously Developed Work}

Currently, I have developed a really simple proof-of-concept macro for Rust typestates.
The macro is based on declarative macros and has several limitations,
the syntax is also distant from the current proposal.
The macro takes a simple typestate specification and expands into several traits and structures.
This project can be found on GitHub \autocite{Duarte2020a}.

\paragraph{Improvement Points.}
As it stands, the project has several possible improvement points to be addressed during the development phase.
Such points are as follows:
\begin{compactitem}
    \item The macro's syntax does not support functions. This implies that it does not support transitions.
    \item The fact that the macro is written using declarative macros makes the underlying code complex and barely readable.
    This is due to the fact that declarative macros are not suited for bigger scale macros.
    \item The macro does not verify state-machine properties, such as the previously enumerated, reachability and termination.
    \item The supported syntax adds new language constructs, such as the \keyword{limited} and \keyword{strict} keywords.
    \item The macro only supports “\emph{phantom}” states, that is, states do not hold any data
    (the name comes from Rust's \keyword{PhantomData}, which is used to implement these kinds of states).
\end{compactitem}

\section{Timeline}

\paragraph{Macro Development.}
This phase consists of the development of the base macro, ensuring the most basic features.
That is the correct transformation of the input source code to typestated Rust, not covering guarantee verification.
\begin{compactitem}
    \item \textbf{Macro Parsing.} Parsing the macro and inner attributes (3 weeks).
    \item \textbf{Code Expansion.} Expanding the macro into new code, adding the actual typestates to the output (3 weeks).
    \item \textbf{Testing.} Ensuring the developed macro is solid and takes care of most use cases, this task includes fixes for possible edge cases (2 weeks).
\end{compactitem}
\paragraph{State Machine Verification Development.}
This phase builds on the previous one, extracting a state machine from the input source code and validating the state machine.
The process between implementation and testing is done iteratively.
\begin{compactitem}
    \item \textbf{Implementation of the algorithms.} Implement the verification algorithms for the state machine, such as the minimality check (5 weeks).
    \item \textbf{Testing.} As discussed, in this phase testing goes hand-in-hand with the implementation, ensuring each algorithm is correct before moving on to the next one (5 weeks).
\end{compactitem}
\paragraph{Usability Checking.}
The final phase is checking the macro for usability, using the macro in a real-world scenario and gather feedback from users.
Identified pain points and other feedback is to be implemented in the macro.
\begin{compactitem}
    \item \textbf{Use case implementation.} Implement a real-life use case, the use case is still to be defined (6 weeks).
    \item \textbf{User survey.} Gather a group of developers to evaluate the macro in a series of tasks, gathering their feedback (3 weeks).
    \item \textbf{User feedback review and integration.} Review the received feedback and address it (3 weeks).
\end{compactitem}
\paragraph{Thesis Writing} is mixed in the end of each phase to account for notes over the work done.
Incrementally building the required body of work and easing the final writing process.

\begin{figure}[h]
    \centering
    \includegraphics[width=\linewidth]{planning.png}
    \caption{Work plan Gantt chart}
\end{figure}