%!TEX root = ../../template.tex
%%%%%%%%%%%%%%%%%%%%%%%%%%%%%%%%%%%%%%%%%%%%%%%%%%%%%%%%%%%%%%%%%%%%
%% abstrac-pt.tex
%% NOVA thesis document file
%%
%% Abstract in Portuguese
%%%%%%%%%%%%%%%%%%%%%%%%%%%%%%%%%%%%%%%%%%%%%%%%%%%%%%%%%%%%%%%%%%%%

Conforme as nossas vidas são cada vez mais dependentes de software,
erros do mesmo têm o potencial de causar problemas significativos.
Prevenir estes erros torna-se uma tarefa prioritária durante o desenvolvimento de sistemas confiáveis.
Erradicar erros por completo é impossível, no entanto é possível eliminar certos conjuntos.

Rust é uma linguagem de programação de sistemas que, por desenho, endereça erros de gestão de memória.
Para o conseguir, a linguagem inclui no sistema de tipos informação sobre o tempo de vida dos objetos,
permitindo assim que o compilador consiga manter uma vista sobre a utilização dos mesmos e verificar por erros de utilização de memória.
Apesar da prevenção de erros de memória ter um papel importante na segurança de software,
existem ainda outras categorias de erros em Rust.
Como o uso incorrecto de interfaces de programação, em que o programador não respeita as restrições impostas pela mesma, resultando numa falha do programa.

\emph{Typestates} elevam o conceito de estado para o sistema de tipos,
permitindo a aplicação das restrições da interface durante a fase de compilação.
Este conceito permite assim aliviar o programador da responsabilidade que é conceptualizar e manter o estado do programa em mente durante o desenvolvimento, prevenindo o mau uso das interfaces.
Apesar de Rust não suportar \emph{typestates} de uma forma natural,
o sistema de tipos permite expressar e validar \emph{typestates}.

O objetivo desta tese é aproximar os \emph{typestates} do Rust em produção,
desenvolvendo uma ferramenta que permite aos programadores tirar partido dos \emph{typestates}.
A mesma ferramenta é também capaz de garantir propriedades extra,
verificando a especificação de \emph{typestates} por erros comuns.

% Palavras-chave do resumo em Português
\begin{keywords}
Tipos comportamentais, \emph{typestates}, meta-programação, macros, Rust
\end{keywords}
% to add an extra black line
