%!TEX root = ../template.tex
\chapter{Designing a DSL in Rust (10/02/2021)}\label{cha:rust-dsl}

\section{Rust Macros}

Just like its predecessors, C \& C++, Rust offers macros as part of the language.
In essence, Rust macros are just like other languages macro's, they generate code before compilation.
However, Rust provides two kinds of macros, \emph{macros by example} (also known as \texttt{macro\_rules!}),
and procedural macros (also known as \texttt{proc-macros}).

\subsection{Macros by example}

The name \emph{macros by example} comes from the way they are written, by pattern matching,
the user writes a pattern that matches a bit on input and the macro will replace it with the defined output.
The most notable differences between C and Rust macros are their hygiene and “\emph{type}”,
C macros are unhygienic and do not distinguish inputs, whereas Rust macros are hygienic
and distinguish between thirteen input kinds (e.g. \texttt{ident}, \texttt{stmt}, \texttt{literal}, etc).

% TODO insert an example
% TODO add more text

\subsection{Proc-macros}

\section{The DSL}
\subsection{Objectives}
What the DSL should achieve
\subsection{Architecture}
How it achieves it